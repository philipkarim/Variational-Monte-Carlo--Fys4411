\documentclass[../main.tex]{subfiles}

\begin{document}

\section{Introduction}
Spectacular advancements in deep learning have been constructed and perfected with time, primarily over the convolutional neural network (CNN) algorithm. The use of CNNs has grown due to its success with a vast array of applications including image classification \cite{He_2016_CVPR}, speech recognition \cite{AbdelHamid2013ExploringCN} and self-driving cars \cite{Wu_2017_CVPR_Workshops}. The aforementioned list of applications is far from complete, but it demonstrates the versatility and the importance of the CNN algorithm. 

The purpose of this work is to explore computational methods of classifying whether a picture contains a person wearing a facemask, incorrectly wearing a facemask or not wearing one at all. In the current climate, this could prove useful in a plethora of ways. An example could be cameras situated at the entrances of a mall, counting the number of people wearing masks and thus calculating the risk of housing a said number of people at the same time.

We will compare the performance of an ordinary logistic regression with a convolutional neural network on the Fashion-MNIST and AI face mask detection dataset. The Fashion-MNIST is a good benchmark dataset which poses a more challenging classification task than the simple MNIST digits data. It will here be used as a proof-of-concept modelling task, and as a learning exercise before we tackle the facemask dataset.
\end{document}